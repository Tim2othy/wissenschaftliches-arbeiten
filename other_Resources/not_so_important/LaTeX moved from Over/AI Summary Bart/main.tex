\documentclass{beamer}
\usepackage{amsmath}
\usepackage{graphicx}

\title{Bayesian Additive Regression Trees (BART)}
\author{Yaoyuan Vincent Tan and Jason Roy}
\date{January 23, 2019}

\begin{document}

\frame{\titlepage}

\begin{frame}
\frametitle{Introduction}
Bayesian Additive Regression Trees (BART) is a flexible prediction model/machine learning approach.
\begin{itemize}
    \item Popularity in various applications: biomarker discovery, causal effect estimation, genomic studies, etc.
    \item Extensions include survival outcomes, multinomial outcomes, semi-continuous outcomes, and more.
\end{itemize}
\end{frame}

\begin{frame}
\frametitle{BART Model Overview}
BART combines multiple regression trees to model complex, non-linear relationships.
\begin{equation}
Y_i = \sum_{j=1}^{m} g(X_i; T_j, M_j) + \epsilon_i
\end{equation}
\begin{itemize}
    \item \(Y_i\): Outcome variable
    \item \(X_i\): Predictor variables
    \item \(T_j, M_j\): Structure and parameters of tree \(j\)
    \item \(\epsilon_i\): Error term
\end{itemize}
\end{frame}

\begin{frame}
\frametitle{Single Regression Tree}
A single regression tree partitions the predictor space and fits simple models within each partition.
\begin{figure}
    \centering
    \includegraphics[width=0.7\linewidth]{tree_example}
    \caption{Example of a single regression tree.}
\end{figure}
\end{frame}

\begin{frame}
\frametitle{Sum of Regression Trees}
BART models use the sum of regression trees to capture complex interactions.
\begin{equation}
Y_i = \sum_{j=1}^{m} g(X_i; T_j, M_j) + \epsilon_i
\end{equation}
\begin{itemize}
    \item Each tree captures different aspects of the data.
    \item Combined result is a flexible model that can approximate non-linear functions.
\end{itemize}
\end{frame}

\begin{frame}
\frametitle{BART Algorithm}
The BART algorithm iteratively updates the trees using Markov Chain Monte Carlo (MCMC).
\begin{enumerate}
    \item Initialize trees to root nodes.
    \item Iteratively update tree structures and parameters.
    \item Ensure convergence to posterior distribution.
\end{enumerate}
\end{frame}

\begin{frame}
\frametitle{Posterior Performance}
BART provides posterior distributions for predictions, allowing for uncertainty quantification.
\begin{itemize}
    \item Example: Posterior performance evaluated using synthetic data.
    \item Real-world application: Predicting Standardized Hospitalization Ratio.
\end{itemize}
\end{frame}

\begin{frame}
\frametitle{Extensions of BART}
\begin{itemize}
    \item Semiparametric BART: Combines parametric and nonparametric components.
    \item Random intercept BART: Models correlated outcomes.
    \item Spatial BART: Addresses statistical matching problems.
    \item Dirichlet Process Mixture BART: Enhances robustness by modeling error terms with a Dirichlet process.
\end{itemize}
\end{frame}

\begin{frame}
\frametitle{Example: Semiparametric BART}
\begin{equation}
Y_i = X_i \beta + \sum_{j=1}^{m} g(X_i; T_j, M_j) + \epsilon_i
\end{equation}
\begin{itemize}
    \item Combines linear predictors with nonparametric regression trees.
    \item Useful for models with both fixed and random effects.
\end{itemize}
\end{frame}

\begin{frame}
\frametitle{Example: Dirichlet Process Mixture BART}
\begin{equation}
\epsilon_i \sim N(a_i, \sigma_i^2), \quad (a_i, \sigma_i^2) \sim D, \quad D \sim DP(D_0, \alpha)
\end{equation}
\begin{itemize}
    \item Models the error term with a Dirichlet process.
    \item Allows for flexible error distributions.
\end{itemize}
\end{frame}

\begin{frame}
\frametitle{Discussion}
BART is a powerful and flexible tool for regression and classification.
\begin{itemize}
    \item Handles complex, non-linear relationships without explicit specification.
    \item Provides a unified framework for various extensions and applications.
    \item Further research can expand BART's applicability and efficiency.
\end{itemize}
\end{frame}

\end{document}
